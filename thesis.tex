% do not change these two lines (this is a hard requirement
% there is one exception: you might replace oneside by twoside in case you deliver 
% the printed version in the accordant format
\documentclass[11pt,titlepage,oneside,openany]{book}
\usepackage{times}


\usepackage{graphicx}
\usepackage{latexsym}
\usepackage{amsmath}
\usepackage{amssymb}

\usepackage{ntheorem}

% \usepackage{paralist}
\usepackage{tabularx}

% this packaes are useful for nice algorithms
\usepackage{algorithm}
\usepackage{algorithmic}

% well, when your work is concerned with definitions, proposition and so on, we suggest this
% feel free to add Corrolary, Theorem or whatever you need
\newtheorem{definition}{Definition}
\newtheorem{proposition}{Proposition}


% its always useful to have some shortcuts (some are specific for algorithms
% if you do not like your formating you can change it here (instead of scanning through the whole text)
\renewcommand{\algorithmiccomment}[1]{\ensuremath{\rhd} \textit{#1}}
\def\MYCALL#1#2{{\small\textsc{#1}}(\textup{#2})}
\def\MYSET#1{\scshape{#1}}
\def\MYAND{\textbf{ and }}
\def\MYOR{\textbf{ or }}
\def\MYNOT{\textbf{ not }}
\def\MYTHROW{\textbf{ throw }}
\def\MYBREAK{\textbf{break }}
\def\MYEXCEPT#1{\scshape{#1}}
\def\MYTO{\textbf{ to }}
\def\MYNIL{\textsc{Nil}}
\def\MYUNKNOWN{ unknown }
% simple stuff (not all of this is used in this examples thesis
\def\INT{{\mathcal I}} % interpretation
\def\ONT{{\mathcal O}} % ontology
\def\SEM{{\mathcal S}} % alignment semantic
\def\ALI{{\mathcal A}} % alignment
\def\USE{{\mathcal U}} % set of unsatisfiable entities
\def\CON{{\mathcal C}} % conflict set
\def\DIA{\Delta} % diagnosis
% mups and mips
\def\MUP{{\mathcal M}} % ontology
\def\MIP{{\mathcal M}} % ontology
% distributed and local entities
\newcommand{\cc}[2]{\mathit{#1}\hspace{-1pt} \# \hspace{-1pt} \mathit{#2}}
\newcommand{\cx}[1]{\mathit{#1}}
% complex stuff
\def\MER#1#2#3#4{#1 \cup_{#3}^{#2} #4} % merged ontology
\def\MUPALL#1#2#3#4#5{\textit{MUPS}_{#1}\left(#2, #3, #4, #5\right)} % the set of all mups for some concept
\def\MIPALL#1#2{\textit{MIPS}_{#1}\left(#2\right)} % the set of all mips





\begin{document}

\pagenumbering{roman}
% lets go for the title page, something like this should be okay
\begin{titlepage}
	\vspace*{2cm}
  \begin{center}
   {\Large Capitalism X\\}
   \vspace{2cm} 
   {Handbook \& Documentation\\}
   \vspace{2cm}
   {presented by\\
    Xy \\
   }
   \vspace{1cm} 
   {submitted to the\\
    Data and Web Science Group\\
    Prof.\ Dr.\ Heiner Stuckenschmidt\\
    University of Mannheim\\} \vspace{2cm}
   {February 2019}
  \end{center}
\end{titlepage} 

% no lets make some add some table of contents
\tableofcontents
\newpage

\listoffigures

\listoftables

% evntuelly you might add something like this
% \listtheorems{definition}
% \listtheorems{proposition}

\newpage
% okay, start new numbering ... here is where it really starts
\pagenumbering{arabic}
\chapter{From a small startup to a global enterprise}
\label{cha:intro}
\section{What to expect from Capitalism X (CapX)}

Business simulation games...Fancy introduction, Target audience

\begin{itemize}
	\item A table can be found in Section \ref{sec:results}. This example (Table \ref{tab:confonly}) is only a suggestion. You are allowed to format your tables in your preferred style.
\end{itemize}

User skills? Thinking economic entrepreneurial spirtit.. blablabla

\emph{Very Important:} Besides being a game-play handbook, this document also addresses the game mechanisms working in the background
 
\section{Player's role}
 
CEO role... 

\section{Story \& Setting}

Goals, Timeframe, Progress

\section{Environment}

General 

\chapter{How to play CapX}
\label{cha:theory}
 This chapter deals with the interface, options, content of the game




\section{Getting started with CapX}
\label{sec:prelim}
Starting screen selections, first steps
\begin{definition}
\label{def:evil}
An entity is evil if it is not a good entity.
\end{definition}

\section{Interface \& Features}
\label{sec:good}

\subsection{Controls}

War Room and what to choose?
\subsection{Financials}
Caaaaash
\subsection{HR}
Text
\subsection{Government}
Text

\section{Build up an empire business}
\label{sec:business}
https://www.youtube.com/watch?v=NisCkxU544c
\subsection{Choose the portfolio}

Description of products

\subsection{Lead the right employees}

Think and ace like a boss

\subsection{Make your company grow}
Stages of the game


\section{Challenges \& Opportunities}
\label{sec:evil}


What to be confronted with 
\chapter{Game Mechanics}
\label{cha:alg}
Technical description of the game, the backend
\section{Features of the Game}
\label{sec:comp-good}

\section{Measurements \& Indicators}
\label{sec:diag}

\section{Products}
\label{sec:products}

\section{Customer Simulation}
\label{sec:customsim}

\section{Linkages \& Interdependencies}
\label{sec:link}

\section{Flowboard}

A flowboard is the best way to document a game’s structure. The term is derived
from a flowchart and a storyboard. Storyboards are a linear series of pictures
used by filmmakers to plan a set of shots. Flowcharts are diagrams used
by programmers for documenting an algorithm. A flowboard combines these
two ideas. Each picture is a sketch or a mockup of the screen, in one specific
gameplay mode or menu. The pictures are connected via arrows that indicate
under what circumstances the transition takes place. (1, p. 57; 3.) The flowboard
does not necessarily need pictures in it to work. A simpler approach (Figure
2) might even be better in some cases.

Many roads lead to Rome. Only one is the best. Choosing the strategy is one thing, making the right decicions another. 

\subsection{Creating uncertainty through random events}

\chapter{Marketing}
\label{cha:exp}

\section{Marketing proposals}
\label{sec:setting}
How to sell to company


\chapter{Conclusion}
\label{cha:conclusion}


\section{Summary}
\label{sec:sum}

\section{Future Work}
\label{sec:future}

\bibliographystyle{plain}
\bibliography{thesis-ref}


\appendix

\chapter{Appendix}
\label{cha:appendix-a}

\newpage


\pagestyle{empty}


\end{document}
