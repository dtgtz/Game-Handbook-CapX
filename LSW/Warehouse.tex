\section{Warehouse Simulation} \label{warehouse_simulation}

%Janine
The storage of finished products and the individual components in CapitalismX is regulated by warehouses. In order to store products and components, the player must first purchase a warehouse, which can either be rented or built by the company itself. Every warehouse has certain characteristics, these include:
\begin{itemize}
    \item A unique construction price of the so called \textit{buildingCost} (\gls{bCW}) of 250.000 CapCoins  \footnote{https://kostencheck.de/hallenbau-kosten} or monthly \textit{rentalCost} (\gls{rCW}) of 10.000 CapCoins \footnote{https://de.statista.com/statistik/daten/studie/303225/umfrage/mietpreis-fuer-lagerflaechen-in-muenchen/}.
    \item A \textit{capacity} (\gls{cW}) of 5000 units. A product corresponds to a unit, regardless of the type of product. Components are counted as half units in order to better account for their proportions. 
    \item \textit{variableStorageCost} (\gls{vSC}) per unit of 5 CapCoins
    \item Monthly \textit{maintenanceCost} (\gls{fCW}) of 1.000 CapCoins
    \item and, if the warehouse was built, a resale value. A warehouse has a \textit{depreciationRate} of 14 years on a straight-line basis \footnote{https://www.betriebsausgabe.de/afa-tabelle/}.
\end{itemize}

After the products are finished, they are taken to a warehouse of the company, no products are sold directly. This means for the production that the \textit{capacity} of all the warehouses together, the \textit{totalCapacity} (\gls{tCWH}), is also the daily maximum total production quantity for the company. The actual possible production quantity or the \textit{freeStorage} (\gls{fSWH}) is calculated by subtracting the number of \textit{storedUnits} (\gls{sU}) from the \textit{totalCapacity}. The exact calculation can be taken from the following formula: 
\begin{equation}
    fSWH = tCWH - sU
\end{equation}

At the same time, the quantity of stored products at the end of the day is also the maximum number of products that can be offered on the market. 

The monthly \textit{costsWarehouse} (\gls{cWH}) are calculated on the basis of the \textit{numberWarehouses} (\gls{nWH}), the \textit{maintenanceCost} and the \textit{rentalCost} for \textit{numberRentedWarehouses} (\gls{nRWH}). The following formula describes the exact calculation of the costs:
\begin{equation}
\label{func:cWH} % costWarehouses
     cWH = nWH \cdot fCW + nRWH \cdot rCW
\end{equation}

Always at the end of the day the sold products are taken out of stock, they do not cause any storage costs. Only \textit{storedUnits} that are not sold and remain in stock overnight incur \textit{variableStorageCost}. This means that the calculation of the \textit{storageCosts} (\gls{sC}) must take place after the sold products have been removed from the warehouse. The daily \textit{storageCosts} are calculated as follows:
\begin{equation}
\label{func:SC} % storageCost
    sC = sU \cdot vSC
\end{equation}

The overall costs incurred in the warehouse department, the so-called \textit{totalWarehouseCost} (\gls{tWC}), are made up of the monthly \textit{costsWarehouse} and the daily \textit{storageCosts}. 