\subsection{Product Support}  \label{product_support_simulation}

%Janine
In addition to the product itself, a product support strategy is also necessary. This strategy should not only consist of maintaining the actual product functionality, but should also include further services for the customer in order to provide him with the greatest possible value with his product \cite{markeset_design_2003}. Additional services could for example be a mobile app or and actively managed community.  
In Capitalism X, the player is offered various options how he wishes to design the product support for his products. Table \ref{Support_types} shows the different services that can be purchased from an external partner, how they influence the \textit{totalSupportTypeQuality} (\gls{tSTQ}) with there \textit{typeQuality} (\gls{sTQ}) and the monthly \textit{typeCost} (\gls{cST}). The variants can be combined by the player as desired. If no option is selected, this automatically means that no support is offered.
Before the company can offer product support to its customers, the player must hire an external support partner. Basically only one partner can be hired at a time, but he can be fired at any time.

\begin{table}[ht]
    \centering
    \begin{tabular}{|l|r|r|}
    \hline
    \textbf{Support type} & \textbf{sTQ} & \textbf{cST} \\
    \hline
    No product support   & -10   & 0    \\
    Online self service  & 0     & 50   \\
    Online support       & 20    & 100  \\
    Telephone support    & 30    & 250  \\
    Store support        & 40    & 400  \\
    Additional services  & 10    & 50   \\     
    \hline
    \end{tabular}
    \caption{Support type influence and costs}
    \label{Support_types}
\end{table}

The \textit{totalSupportTypeQuality} is calculated from the various support types offered; this is simply the sum of the support types. The quality of the \textit{totalSupportQuality} (\gls{tSQ}) of the company is influenced not only by the offered support types but also by the dedicated support company, more precisely its \textit{qualityIndex} (\gls{qSP}). The calculation is done as follows:
\begin{equation}
\label{func:totalProductSupport}
\begin{aligned}
    tSQ = &
    \begin{cases}
        0.4 \cdot qSP + 0.6 \cdot tSTQ & \text{if } qSP \leq  50\\
        0.3 \cdot qSP + 0.7 \cdot tSTQ & \text{otherwise}
    \end{cases}
\end{aligned}
\end{equation}

The \textit{totalSupportCost} (\gls{tSC}) are calculated from the sum of the monthly \textit{typeCost} and the \textit{contractualCost} (\gls{cSP}) for cooperation with the partner company.
\begin{center}
\begin{equation}
tSC=\sum_{st \in ST}{cST} + cSP
\label{eq:totalSupportCost}
\end{equation}
where \\
\textit{ST} = Set of offered support types 
\end{center}