\subsection{Logistic} \label{logistic_simulation}

%Janine
The delivery of the products always takes place at the end of the day directly from the warehouse and follows the first in first out principle. Basically there are three ways to manage the logistic of the company: 
\begin{itemize}
    \item The company has its own logistics fleet
    \item The company has outsourced the logistics to an external partner
    \item The company simply sends products by mail
\end{itemize}

The player can combine these variants in the way he considers best for his company. However, there are four logical approaches:
\begin{enumerate}
    \item The player decides to send all his products by post.
    \item The player decides to completely outsource the logistics and signs a contract with an external logistics partner.
    \item The player exclusively employs his own logistics fleet and ships products that exceed his delivery capacity at high cost by post.
    \item The player has his own fleet and delivers products which exceed his capacity by an external logistics partner.
\end{enumerate}
	
In the following sections, the functionality of the individual variants and the costs incurred are explained in detail.


\subsubsection{Internal logistic fleet}
The internal logistics fleet consists only of purchased trucks. The quality of the company's own logistics is calculated based on the characteristics of the single trucks. Every truck has certain characteristics, these consist of:
\begin{itemize}
    \item A \textit{capacity} of products that can be transported, this is the same for all trucks and amounts to 1,000 packages. For reasons of simplification, we do not distinguish the size of individual products.  Each product is equivalent to one package. 
    \item An \textit{ecoIndexTruck} ( \gls{eIT}) that expresses how environmentally friendly the vehicle is.
    \item A \textit{qualityIndexTruck} (\gls{qIT}) that expresses the quality of the transport.
    \item A \textit{purchasePriceTruck} (\gls{pPT})that depends on the quality and environmental friendliness of the truck. The base price for a truck is 100,000 CapCoins 
    \footnote{https://www.motor-talk.de/forum/wie-teuer-ist-ein-neuer-lkw-t1631664.html}.
    \item Monthly maintenance costs corresponding to 0,1\% of the initial purchase price divided by 12 months. These monthly cost are called \textit{fixTruckCost} (\gls{fTC})
    \item And a \textit{depreciationRateTruck} that is linear and amounts to 9 years for each truck \footnote{https://www.betriebsausgabe.de/afa-tabelle/}.
\end{itemize}

As already described, \textit{qualityIndexTruck} and \textit{ecoIndexTruck} influence the initial \textit{purchasePriceTruck} of a truck, which is 10,000 CapCoins. In order to give the player more freedom for decisions, there are basically six different truck models available for purchase, table \ref{Truck_characteristics} shows how they are composed in detail.

The \textit{qualityIndexTruck} and the \textit{ecoIndexTruck}, are set randomly in the given ranges.The \textit{purchasePriceTruck} for the respective truck and the \textit{fixCostsDelivery} (\gls{fCD}) are calculated by multiplying the base price with the factors. These factors are also randomly assigned in a certain value range. Cause of the fact that the intervals of the individual models overlap, it can happen that a more favourable model has better properties than a more expensive one.

\begin{table}[ht]
    \centering
    \begin{tabular}{|l|r|r|r|}
    \hline
    qIT & eIT & Influence pPT & influence fCD \\
    60 - 100      & 60 - 100   & 1.1 - 1.3   & 0.6 - 0.8       \\
    60 - 100      & 0 - 40     & 0.9 - 1.1   & 0.8 - 1.1       \\
    30 - 70       & 60 - 100   & 1.0 - 1.2   & 0.7 - 0.9       \\
    30 - 70       & 30 - 70    & 0.8 - 1.1   & 0.9 - 1.1       \\
    0 - 40        & 30 - 70    & 0.7 - 0.9   & 1.0 - 1.2       \\
    0 - 40        & 0 - 40     & 0.6 - 0.8   & 1.1 - 1.3       \\
    \hline
    \end{tabular}
    \caption{Truck characteristics and calculations}
    \label{Truck_characteristics}
\end{table}

The total capacity thus results from the \textit{numberTrucks} (\gls{nT}) in the company's own logistics fleet and is called \textit{capacityFleet} (\gls{cF}). 
\begin{equation}
\label{func:capacityFleet}
    cF = 1000 \cdot nT
\end{equation}

% describe totalEcoIndexTruck

totalEcoIndexTruck (\gls{tEIT})= \frac{\sum eIT }{nT}

The overall quality internal logistics, the so-called \textit{internalLogisticIndex} (\gls{iLI}), is calculated on the basis of the \textit{qualityIndexTruck} and the \textit{ecoIndexTruck} of the individual trucks.
\begin{equation}
\label{func:internalLogisticIndex}
\begin{aligned}
    Eco Index Fleet (\gls{eF}) = \sum \frac{eIT}{nT} \\
    Quality Index Fleet (\gls{qF}) = \sum \frac{qIT}{nT}\\
    Internal Logistic Index (\gls{iLI}) = eF \cdot 0,2 + qF \cdot 0,8 \\
\end{aligned}
\end{equation}

Of course, the delivery of the products entails costs for the company. These consist of the \textit{fixCostDelivery} and the \textit{variableCostsDelivery} per delivered package (\gls{vCD}). 
A delivery by truck in principle costs 2,000 CapCoins. Like the purchase price, the fixed delivery costs also vary on the basis of \textit{qualityIndexTruck} and \textit{ecoIndexTruck}. The \textit{variableCostsDelivery} per product are the same for each truck and amount to 2 CapCoins.

The cost of a delivery thus depends on the total number of \textit{deliverdProduct} (\gls{dP}). 
The case that the number of \textit{deliverdProduct} exceeds the \textit{capacityFleet} must be considered separately. The additional \textit{costsExternalDelivery} (\gls{cED}) can either be caused by a contractual external logistics partner or by sending the packages by post. How the costs are composed in detail is described in the following sections. \\
The following formula explains how the total totalDeliveryCost (\gls{tDC}) are calculated. 
\begin{equation}
\label{func:deliveryCost}
\begin{aligned}
tDC = 
\begin{cases}
    (( \frac{dP}{1000} ) \cdot fCD) + ( dP \cdot vCD) & \text{if} dP = cF\\
    (\lceil \frac{dP}{1000} \rceil \cdot fCD ) + ( dP \cdot vCD) & \text{if} dP < cF\\
    ( nT \cdot fCD ) + ( nT \cdot1000 \cdot vCD ) + cED & \text{otherwise}\\
\end{cases}
\end{aligned}
\end{equation}

As already mentioned, the annual \textit{maintenanceCosts}  per truck amount to 10\% of the purchase value. This is not an increase in the value of the truck, it is simply repairs and registration costs. The sum of these cost for all the trucks in the logistic fleet are called \textit{totalTruckCost} (\gls{tTC}). The monthly costs for the entire logistics fleet are calculated as follows. 
\begin{equation}
    tTC = \sum fTC
\end{equation}


\subsubsection{External logistic Partner}
In addition to his own logistics fleet, the player has the option of forming a contract with an external logistics partner. It is possible to outsource the complete product delivery to this partner or only products that exceed the delivery capacity of the own fleet. Basically, only one partner can be contracted at a time, but he can be fired at any time and replaced by another partner available on the market. 

The external logistics partners also have certain characteristics that influence the quality of the delivery. These are the attributes of an external logistic partner:  

\begin{itemize}
    \item An \textit{ecoIndexPartner} (\gls{eIP}) that expresses how environmentally friendly the external partner is working (0-100)
    \item A \textit{qualityIndexPartner} (\gls{qIP}), which expresses how good the quality of the partner's delivery is (0 -100)
    \item A \textit{reliabilityIndexPartner} (\gls{rIP}), which expresses how reliable the partner is
    \item monthly fixed \textit{contractualCost} (\gls{cCLP}) for the provision of the service, which are based on the quality, environmental friendliness and reliability of the external logistics partner. The basic price is 1,000 CapCoins 
    \item And contractually agreed \textit{variableDeliveryCost} (\gls{vCDE}) per delivered product.These are generally 5 CapCoins, but this price also depends on the characteristics of the service provider.
\end{itemize}

The table \ref{External_logistic_partner_characteristics} shows how exactly the costs are related to the properties and which characteristics can occur in the game. Within the specified value ranges, the values are assigned randomly. The contractual costs and the variable costs are calculated by the initial price multiplied by the factor that depends on the partner characteristics.

\begin{table}[ht]
    \centering
    \begin{tabular}{|r|r|r|r|r|}
    \hline
    qIP & eIP & rIP & influence tCCLP & influence vCDE \\
    60 - 100      & 60 - 100   & 60 - 100  & 1.1 - 1.3    & 0.7 - 0.9     \\
    60 - 100      & 30 - 70    & 30 - 70   & 1.0 - 1.2    & 0.8 - 1.0     \\
    60 - 100      & 0 - 40     & 60 - 100  & 1.0 - 1.2    & 0.8 - 1.0     \\
    30 - 70       & 60 - 100   & 0 - 40    & 0.9 - 1.1    & 0.9 - 1.1     \\
    30 - 70       & 30 - 70    & 30 - 70   & 0.9 - 1.1    & 0.9 - 1.1     \\
    30 - 70       & 0 - 40     & 60 - 100  & 0.9 - 1.1    & 0.9 - 1.1     \\
    0 - 40        & 60 - 100   & 30 - 70   & 0.9 - 1.1    & 0.9 - 1.1     \\
    0 - 40        & 30 - 70    & 30 - 70   & 0.8 - 1.0    & 1.0 - 1.2     \\
    0 - 40        & 0 - 40     & 0 - 40    & 0.7 - 0.9    & 1.1 - 1.3     \\
    \hline
    \end{tabular}
    \caption{external Logistic Partner characteristics and cost calculations}
    \label{External_logistic_partner_characteristics}
\end{table}

The total influence of an external partner company is measured by the \textit{externalLogisticsIndex} (\gls{eLI}). Depending on how high its reliability is, different priorities are set in the calculation. The following formula shows the exact calculation: 
\begin{equation}
\label{func:externalLogisticIndex}
\begin{aligned}
eLI = 
    \begin{cases}
     (rIP\cdot0,5 + 0,5\cdot(qIP\cdot0,8 + eIP\cdot0,2)) & \text{if} ~rIP \leq 40\\
     (rIP\cdot0,4 + 0,6\cdot(qIP\cdot0,8 + eIP\cdot0,2)) & \text{otherwise} \\
    \end{cases}
\end{aligned}
\end{equation}

If the player decides to deliver all products by an external logistics partner, the \textit{costsExternalDelivery} of the company are calculated as follows:
\begin{equation}
\label{func:costExternalDelivery_all}
   cED = dP \cdot vCDE
\end{equation}

If the company has its own logistics fleet, the products are initially delivered via the internal fleet. However, if the company's own capacities are not sufficient, the remaining products are taken over by the external partner. In this case, the \textit{costsExternalDelivery} are calculated as follows: 
\begin{equation}
\label{func:costExternalDelivery_partly}
    cED = ( dP – cF ) \cdot vCDE
\end{equation}

If the company has not engaged an external partner and exceeds the \textit{capacityFleet}, the remaining products will be sent by post. For each package there will be a \textit{shippingFee} (\gls{sF}) of 15 CapCoins. Also in this case delivery quality and environmental friendliness are included. The \textit{qualityIndexPartner} is basically 60 and the \textit{ecoIndexPartner} 40. The \textit{costsExternalDelivery} are calculated as follows:  
\begin{equation}
\label{func:costExternalDelivery_post}
    cED = ( dP \ - \ cF ) \cdot sF
\end{equation}

Together with the \textit{totalTruckCost} the \textit{contractualCost} make up the monthly fixed costs of the logistics department. These are called \textit{costLogistic} (\gls{cL}). 

The overall quality of the company's logistics as perceived by its customers is expressed by the \textit{logisticIndex} (\gls{lI}). This includes both the internal logistics and the external logistics in proportion to the products delivered. The following formula shows the exact calculation of the \textit{logisticIndex}:
\begin{equation}
\label{func:logisticIndex}
\begin{aligned}
lI = 
\begin{cases}
     ((( dP \ - \ cF) \cdot eLI ) + ( cF \cdot iLI )) & \text{if} ~dP > cF \\
     cF \cdot iLI & \text{if} ~dP = cF\\
     dP \cdot iLI & \text{otherwise} \\
\end{cases}
\end{aligned}
\end{equation}

The \textit{totalLogisticCost} (\gls{tLC}) consists of the monthly \textit{logisticCosts} and the daily \textit{deliveryCost}. To determine the \textit{totalLogisticCost} on a daily basis, the monthly \textit{logisticCost} is divided by 30. 
\begin{equation}
    tLC = (\frac{lC}{30}) + dC
\end{equation}
