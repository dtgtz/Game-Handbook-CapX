\subsection{Logistic}
%Janine
The delivery of the products always takes place at the end of the day directly from the warehouse. Basically there are three ways to manage the logistic of the company: 
\begin{itemize}
    \item The company has its own logistics fleet
    \item The company has outsourced the logistics to an external partner
    \item The company simply sends products by mail
\end{itemize}

The player can combine these variants in the way he considers best for his company. Basically, there are four logical approaches:
\begin{enumerate}
    \item The player decides to send all his products by post 
    \item The player decides to completely outsource the logistics and signs a contract with an external logistics partner
    \item The player exclusively employs his own logistics fleet and ships products that exceed his delivery capacity at high cost by post
    \item The player has his own fleet and delivers products which exceed his capacity by an external logistics partner
\end{enumerate}
	
In the following sections, the functionality of the individual variants and the costs incurred are explained in detail.


\subsubsection{Internal logistic fleet}
The internal logistics fleet consists of purchased trucks and the company's logistics employees. The quality of the company's own logistics is calculated on the basis of their characteristics and motivation.

Every truck has certain characteristics, these consist of:
\begin{itemize}
    \item A capacity of products that can be transported, this is the same for all trucks and amounts to 1000 packages. For reasons of simplification, we do not distinguish the size of individual products.  Each product is equivalent to one package. 
    \item An environmental index (eIT) that expresses how environmentally friendly the vehicle is.
    \item A quality index that expresses the quality of the transport.
    \item A purchase price that depends on the quality and environmental friendliness of the truck.
    \item Monthly maintenance costs corresponding to (TODO)\% of the initial purchase price divided by 12 months (REFERENCE)
    \item And a depreciation rate that is linear and amounts to 9 years for each truck (REFERENCE) 
\end{itemize}

As already described, quality and environmental friendliness influence the initial purchase price of a truck, which is 10.000 Cap Coins. In order to give the player more freedom for decisions, there are basically six different truck models available for purchase, table (REF) shows how they are composed in detail. 

The quality and the environmental index, are set randomly in the given ranges.The purchase price for the respective truck and the fixed delivery costs are calculated from the based on the base values times the factors which are also randomly assigned in a certain value range. Cause of the fact that the intervals of the individual models overlap, it can happen that a more favourable model has better properties than a more expensive one.


\begin{table}[ht]
    \centering
    \begin{tabular}{|l|r|r|r|}
    \hline
    Quality Index & Eco Index & Influence on base price & influence on fix delivery cost \\
    60 - 100      & 60 - 100   & 1.1 - 1.3   & 0.6 - 0.8       \\
    60 - 100      & 0 - 40     & 0.9 - 1.1   & 0.8 - 1.1       \\
    30 - 70       & 60 - 100   & 1.0 - 1.2   & 0.7 - 0.9       \\
    30 - 70       & 30 - 70    & 0.8 - 1.1   & 0.9 - 1.1       \\
    0 - 40        & 30 - 70    & 0.7 - 0.9   & 1.0 - 1.2       \\
    0 - 40        & 0 - 40     & 0.6 - 0.8   & 1.1 - 1.3       \\
    \hline
    \end{tabular}
    \caption{Truck characteristics and calculations}
    \label{Truck_characteristics}
\end{table}

The total capacity thus results from the number of trucks (NT) in the company's own logistics fleet. 
\begin{equation}
    Capacity internal logistics fleet (cF) = 1000 * NT
\end{equation}

The overall quality internal logistics, the so-called Internal Logistic Index (iLI), is calculated on the basis of the quality and environmental friendliness of the individual trucks.  
\begin{equation}
\begin{aligned}
    Eco Index Fleet (eF) = Sum (eIT) / NT \\
    Quality Index Fleet (qF) = Sum (qIT) / NT \\
    Internal Logistic Index (iLI) = eF * 0,2 + qF * 0,8 \\
\end{aligned}
\end{equation}

Of course, the delivery of the products entails costs for the company. These consist of the fixed costs for the ride of a truck (FCD) and the variable costs per delivered package (VCD). 
A delivery by truck in principle costs (TODO) Cap Coins. Like the purchase price, the fixed delivery costs also vary on the basis of quality and eco index. The variable delivery costs per product are the same for each truck and amount to (TODO) Cap Coins.

The cost of a delivery thus depends on the total number of products to be delivered (DP). 
The case that the number of products to be delivered exceeds the total capacity of the internal delivery fleet must be considered separately. The additional costs incurred by an external delivery (CED) can either be caused by a contractual external logistics partner or by sending the packages by post. How these costs are composed in detail is described in the following sections. 
\begin{equation}
\begin{aligned}
Delivery cost (DC) = if ( DP = CF ) { (( DP / 1000) * FCD) + ( DP * VCD) } \\
elseif ( DP < CF ) { (round up to next int ( DP / 1000) * FCD ) + ( DP * VCD) } \\
else { ( NT * FCD ) + ( NT *1000 * VCD ) + CED }
\end{aligned}
\end{equation}

\subsubsection{External logistic Partner}
In addition to his own logistics fleet, the player has the option of forming a contract with an external logistics partner. It is possible to outsource the complete product delivery to this partner or only products that exceed the delivery capacity of the own fleet. Basically, only one partner can be contracted at a time, but he can be fired at any time and replaced by another partner available on the market. 
The external logistics partners also have certain characteristics that influence the quality of the delivery: 

\begin{itemize}
    \item An eco index (EIP) that expresses how environmentally friendly the external partner is working (0-100)
    \item A Quality Index (QIP), which expresses how good the quality of the partner's delivery is (0 -100)
    \item A reliability index (RIP), which expresses how reliable the partner is
    \item Contractually agreed monthly fixed costs (cFCEP) for the provision of the service, which are based on the quality, environmental friendliness and reliability of the external logistics partner. The basic price is (TODO) Cap Coins 
    \item And contractually agreed variable delivery costs (vCDE) per delivered product.These are generally (TODO) cap coins, but this price also depends on the characteristics of the service provider
\end{itemize}

The table \ref{External_logistic_partner_characteristics} shows how exactly the costs are related to the properties and which characteristics can occur in the game. Within the specified value ranges, the values are assigned randomly. 
The contractual costs and the variable costs are calculated by the initial price multiplied by the factor that depends on the partner characteristics.

\begin{table}[ht]
    \centering
    \begin{tabular}{|l|r|r|r|r|}
    \hline
    Quality Index & Eco Index & Reliability Index & influence cFCEP & influence vCDE \\
    60 - 100      & 60 - 100   & 60 - 100  & 1.1 - 1.3    & 0.7 - 0.9     \\
    60 - 100      & 30 - 70    & 30 - 70   & 1.0 - 1.2    & 0.8 - 1.0     \\
    60 - 100      & 0 - 40     & 60 - 100  & 1.0 - 1.2    & 0.8 - 1.0     \\
    30 - 70       & 60 - 100   & 0 - 40    & 0.9 - 1.1    & 0.9 - 1.1     \\
    30 - 70       & 30 - 70    & 30 - 70   & 0.9 - 1.1    & 0.9 - 1.1     \\
    30 - 70       & 0 - 40     & 60 - 100  & 0.9 - 1.1    & 0.9 - 1.1     \\
    0 - 40        & 60 - 100   & 30 - 70   & 0.9 - 1.1    & 0.9 - 1.1     \\
    0 - 40        & 30 - 70    & 30 - 70   & 0.8 - 1.0    & 1.0 - 1.2     \\
    0 - 40        & 0 - 40     & 0 - 40    & 0.7 - 0.9    & 1.1 - 1.3     \\
    \hline
    \end{tabular}
    \caption{external Logistic Partner characteristics and cost calculations}
    \label{External_logistic_partner_characteristics}
\end{table}

The total influence of an external partner company is measured by the External Logistics Index (ELI). Depending on how high its reliability is, different priorities are set in the calculation. The following formula shows the exact calculation: 
\begin{equation}
\begin{aligned}
    if (rLP ≤ 40) eLI = { (rLP*0,5 + 0,5*(qLP*0,8 + eLP*0,2)) } \\
    else eLI = { (rLP*0,4 + 0,6*(qLP*0,8 + eLP*0,2)) }
\end{aligned}
\end{equation}

If the player decides to delivered all products (dP) by an external logistics partner, the daily costs for the company are calculated as follows:
\begin{equation}
    Cost external delivery (cED) = dP * vCDE
\end{equation}

If the company has its own logistics fleet, the products are initially delivered via it. However, if the company's own capacities are not sufficient, the remaining products are taken over by the external partner. In this case, the costs are calculated as follows: 
\begin{equation}
    CED = ( DP – CF ) * VCDE
\end{equation}

If the company has not engaged an external partner and exceeds the delivery capacity, the remaining products will be sent by post. For each package there will be a shipping fee (SF) of ... Cap Coins. Also in this case delivery quality and environmental friendliness are included. The quality index (qIP) is basically (TODO) and the eco index (eIP) (TODO). The external delivery costs are calculated as follows:  
\begin{equation}
    cED = ( dP – cF ) * sF
\end{equation}

The overall quality of the company's logistics as perceived by its customers is expressed by the logistics index. This includes both the internal logistics and the external logistics in proportion to the products delivered. The following formula shows the exact calculation of the Logistic Index (lI):
\begin{equation}
\begin{aligned}
    if ( dP > cF ) lI = { ((( dP – cF) * eLI ) + ( cF * iLI )) / dP} \\
    elseif ( dP = cF ) lI = { cF * iLI } \\
    else lI = { dP * iLI }  
\end{aligned}
\end{equation}