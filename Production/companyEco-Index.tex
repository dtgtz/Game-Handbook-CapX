\section{Company Eco-Index}
\label{sec:compEco-idex}
In this simulation, we measure all events related to the environment, with the company's \textit{ecoIndex}. This metric allows to view the current environmental status of the company. There is a $5$ points scale in which the company is evaluated, which we are going to refer to in this chapter due to simplicity. The scale is explained in the Table \ref{table:eco-index-Company} below and it references the standard $0-100$ points range. \\


\begin{table}[ht]
\centering
\begin{tabular}{|c|c|c|c|}
\hline
 Index & Color & Quality of Index & Range \\
\hline
 5 & \cellcolor[HTML]{228b22}Green & Good & 80-100\\ \hline
 4 & \cellcolor[HTML]{ffff00}Yellow & Moderate & 60-79\\  \hline
 3 & \cellcolor[HTML]{ffd700}Orange & Unhealthy & 40-59\\ \hline
 2 & \cellcolor[HTML]{ff0000}Red & Very Unhealthy & 20-39 \\ \hline
 1 & \cellcolor[HTML]{a52a2a}Maroon & Hazardous & 0-19\\
\hline
\end{tabular}
\caption{Company Eco-Index}
\label{table:eco-index-Company}
\end{table}

The quality of index presented in the table above, has several repercussions on different factors that influence the company. Namely these factors are \textit{tax}, \textit{employeeSatisfaction}, \textit{companyImage} and external events. Taxes are imposed when the index first starts deteriorating and reaches level $4$. For each lower level, the tax is increased as to further penalize the company for its environmental neglect. The impact on \textit{employeeSatisfaction} is perceived as good in the case of level $5$ and $4$, and it gets low on the consecutive levels. The \textit{companyImage} index reacts in the same manner, to the decrease in \textit{ecoIndex}.

Events start at level $3$ where new eco-laws are imposed, which means the government is putting the company in alert, regarding their negative environmental impact. This alert is represented by the setting of additional fines, to further penalize the company in monetary terms. In level $2$, these eco-laws become harsher, which means that very large taxes and fines are imposed. On level $1$, there is an important event, as the company has become hazardous for the environment, the government shuts down the company and that means "Game Over" for our player. \\
 These implications can be found in a more visual form in Table \ref{table:eco-index-CompanyE}.

\begin{table}[ht]
\centering
\begin{tabular}{|c|c|c|c|c|}
\hline
 Index & Tax & \begin{tabular}{@{}c@{}}Employee \\ Satisfaction\end{tabular} & \begin{tabular}{@{}c@{}}Company \\ Image\end{tabular} & Event \\
\hline
 5 & - & good & good & - \\ \hline
 4 & low  & good & good & - \\  \hline
 3 & high & low &low & new eco-law\\ \hline
 2 & very high & low &low & harsh eco-law \\ \hline
 1 & - & - & - & \begin{tabular}{@{}c@{}}Government \\ shut down company\end{tabular}\\
\hline
\end{tabular}
\caption{Production Technology effect on Company}
\label{table:eco-index-CompanyE}
\end{table}

According to the actions taken by management, the company in our simulation can regulate or deteriorate its position in the scale, by acting in the ways mentioned below. \\
The \textit{ecoIndex} lowers $1$ point down the scale if:

\begin{itemize}
	\item Old machinery is operating in the production department. After being used for a time-span of more than $5$ years machinery gets less effective and it pollutes more. If not maintained or upgraded the low machinery level will cause the Eco-Index of the company to deteriorate.

\item Logistic vehicles with low regard for the environment are used. This is measured by the average logistic eco-index. If it is lower than level $3$, the company's \textit{ecoIndex} will drop by one level.
\item Each time-span of $3$ years, if the company does not meet with pollution regulatory laws, the company's \textit{ecoIndex} lowers automatically and gives a notification to the player.
\end{itemize}

The \textit{ecoIndex} lowers $2$ points down the scale if: 
\begin{itemize}
	\item There is an incident inside the company that results in an environmental disaster. There are always unpredictable incidents that may occur while operating a production facility. This is handled by making random events that will trigger this \textit{ecoIndex} decrease.These incidents also include fires that may burn a part of your company, the breach and spread of a dangerous bio-chemical element, etc.
\end{itemize}

A company can also improve its \textit{ecoIndex}, by choosing one of these options which are also part of the marketing simulation in chapter \ref{company_image} : 

\begin{itemize}
\item Promote environmental friendly supplier
\item Promote environmental friendly production
\item Green marketing campaign
\end{itemize}
Referring to table \ref{calculation_CI} in the Marketing simulation, where these ecological campaigns are defined, the total points a player can gather from conducting these three campaigns is 11.5. They will affect the \textit{ecoIndex}, only if these campaigns amount to a total number of points which is $9$ or larger. If the total number of poinsts is between $9$ and $10$ the \textit{ecoIndex} of the company will increase by one level, and if the number of points provided by these campaigns is larger than $10$ the \textit{ecoIndex} increases by two levels. It is worth mentioning that these ecological campaigns, together with all the other campaigns, provided in table \ref{calculation_CI} impact the \textit{companyImage}.

An important function that drives the \textit{ecoIndex} is defining the \textit{ecoCost}. This cost measures the amount of CapCoins that the company is due to pay to the government if it has a low index. \\ 
\begin{equation}
eco–Tax= eco–Flat tax+ additionalEcoTax
\label{eq:eco-tax}
\end{equation}
\begin{equation}
ecoCost= ecoTax - (pT + component \; eI)\times ecoRefund 
\label{eq:eco-cost}
\end{equation}
\begin{center}
with\\
	pT=Production Technology\\
	component \gls{eI}= component ecoIndex\\
\end{center}

In Equation \ref{eq:eco-cost}, component eI refers to the component  \textit{ecoIndex}, as the components are purchased from the supplier, they have their own \textit{ecoIndex}.\\
The \textit{ecoTax} in equation \ref{eq:eco-tax} is composed of the \textit{ecoFlatTax} and the \textit{additionalEcoTax} set by the government, if the index is lower than $4$, as shown in Table \ref{table:eco-index-CompanyE}.
The \textit{ecoFlatTax} is a fixed tax amounting to $10,000cc$. The \textit{ecoCost} can also be $0$ if the company has a quarterly  \textit{ecoIndex} of $5$. The  \textit{ecoRefund} is a government refund of $1,000cc$, that makes it possible to pay no \textit{ecoTax} at all, in the case that the \textit{ecoIndex} is at its highest. 