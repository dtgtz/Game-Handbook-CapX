\subsection{Financials}
Nike, Salih\\

Financial actions occur throughout the whole company as it can be seen as the engine that keeps the whole company running. The success of your company depends on your financial strength. A healthy balance between income and expenses ensures the competitiveness of your company. Therefore, the organization of finances is an essential part of management. The question of raising capital must be defined in order to have a good mix between debt and equity.\\

You can choose from a variety of ways to raise capital. A distinction is made between internal and external financing vs. equity and debt financing.
In CapitalismX the main source of capital should be the internal financing from your goods sold. You can also depreciate aged and no more used machines, trucks and buildings and sell them for the balance value. However, also external financing in the form of debt financing is possible. In this case, you as the CEO have the possibility to loan money from the bank.\\

[Steffen insert Investment manual]\\
Apart from that, you might invest some money into innovations on the stock market, which is similar to the well-known DAX stock market. As this may be a risky undertaking, due to the regular oscillations on this market, a good idea is to spend only a reasonable amount of money there. \\


 The entry point for the financial actions is the finance dashboard. You may find it by clicking on the finance button [screenshot] from the navigation bar at the top left. The dashboard gives you a detailed overview about several areas at first glance:
\begin{itemize}
    \item Cashflow, which is basically your profit and loss statement displaying revenues, expenses, and profit on a quarterly basis.
    \item Statistics, showing the development of salaries and total sales in visualized graphs.
    \item Company information, divided into cash, assets, liabilities and company net worth. This view gives you information about the current liquidity of your company.
    \item Bank, providing the possibility to request a loan from the bank or to get an overview about current loans.
    \item Investments, where the you might invest some money in the stock market.
    \item Product performance, giving an overview about product specific key performance indicators, such as the sum of material costs per product.
\end{itemize}
[SCREENSHOT IN APPENDIX WITH REFERENCE???] The cashflow, statistics, company information and product performance are solely descriptive information, that is generated from the actual settings defined by you as the player or by your company’s performance. The goal of these areas is to provide an overview about the general financial performance status of the company and should give you hints about adapting some parameters. [DESCRIBE MORE IN DETAIL, e.g. company net worth is ebit with added loans and substracted taxes and interests i.e. most important financial key figure bla???] However, the bank and investments area require input from your side, which you can provide as needed. \\

While you are building an empire, it can happen that you suffer financial bottlenecks. In order to avoid or react to bottlenecks you might want to make use of debt financing, hence you have the possibility to borrow money from your bank. \\

You can visit the bank using the Finance Dashboard. Here you can request credits, but you can also see an overview of credits you have already received.
The list of credits you have already received gives you a detailed overview of repayments. Here you will find a list of the loan amount, the monthly or annual interest and retirement as well as your remaining debt. The agreed repayment plan is automatically booked from your bank account in accordance with the previously determined conditions.
If you are in financial trouble or planning a major investment, you can request a loan from your bank. The bank will provide you with a choice of short, medium, and long term loans, which you can accept or decline. You should note that the bank always uses your company value as a basis for assessment and takes this into account when granting loans or reject your request. \\

Pricing and selling of goods might appear as a financial task to you, but in fact, you will find information about these topics in the manufacturing view.\\ 

%Pricing and selling -- put into production view \\
%pricing: user sees cumulated costs per product incl. employee costs (LIVJAS formel) and may decide on a profit margin to add to the costs. He gets some hints and the value addded as percent. (explain UI part in 4.x chapter).

