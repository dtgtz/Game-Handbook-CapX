\subsection{Investments}
\label{sec:investments_simulation}

 Investments in CapitalismX are a way to allocate your residual cash to assets with an expected profit. However, each investment is risky and positive return isn't guaranteed. Thus, your liquidity may suffer if you put too much money into investments.
  
  We decided for three classes of investments with different expected returns and assume higher risk for higher return:
\begin{itemize}
	\item Real estate fund
	\item Stocks (index fund)
	\item Venture capital fund
\end{itemize}

Although return distributions are negatively skewed and characterized by an excess kurtosis \cite{ANDERSEN200143}, a normal distribution can be used as an approximation \cite{doi:10.1080/01621459.1972.10481297}. In 2017, the expected return for real estate in the US is about $7\%$ \footnote{https://www.msci.com/documents/10199/f3d1af7c-6069-6e60-a818-71ca9c45e85c}. The compounded annual return of the S\&P since its introduction in 1965 is approximately $10\%$ \footnote{http://www.berkshirehathaway.com/letters/2017ltr.pdf}. Venture capital funds in the US have an expected return of $14.2\%$ \footnote{http://www.eif.org/news\_centre/publications/eif\_wp\_41.pdf}.

In our game, the investment worth adjusts daily. Therefore, we take the geometric mean of the average yearly return $\overline{\mu_y}$ to estimate the expected daily return:
\begin{equation}
	\overline{\mu_d} = \sqrt[365]{\frac{1 + \overline{\mu_y}}{1}}
\end{equation}

For our game, we use the return volatility as a representative of the risk of an investment. In an efficient market, a higher risk has to correspond to a higher return. As we are not able to model more complex anomalies like volatility clustering we choose the daily standard deviation $\sigma$ by testing different values instead of relying on historical data in order to allow reasonable game mechanics. However, as a reference point, the yearly standard deviation of the S\&P 500 was 19.7\% \footnote{https://seekingalpha.com/instablog/605212-robert-allan-schwartz/4831186-annual-returns-s-and-p-500-1928-2015}. 

We set the yearly standard deviation $\sigma_T$. Thus, we need to calculate the standard deviation of a single day, which we assume is the same in each period. \textit{square-root-of-time-rule} gives the standard deviation of the entire time frame $T$.
\begin{equation}
    \sigma_T = \sigma*\sqrt{T}
\end{equation}
Thus, we can calculate the daily return from the yearly returns by applying algebraic transformations:
\begin{equation}
    \sigma = \frac{\sigma_T}{\sqrt{T}}
\end{equation}
In the end, our daily return is generated by a Gaussian distribution:
\begin{equation}
	\mu_d \sim \mathcal{N}(\overline{\mu_d},\,\sigma^{2})\,.
\end{equation}