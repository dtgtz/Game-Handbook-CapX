\section{Customer Simulation} 
\label{sec:customsim}

\subsection{Customer Satisfaction}
\label{customerSatisfaction}
In this game, the customer simulation represents the customers' buying decision. How exactly the customer interest is calculated is discussed in detail in chapter \ref{demandCalc}. The prerequisite for this is the calculation of the \textit{customerSatisfaction} (\textit{\gls{cS}}), which will be discussed in more detail in this chapter.

As in reality, the \textit{customerSatisfaction} plays an important role in terms of the number of products sold and thus in terms of sales. \cite{deptolla_effects_2004} In CapitalismX, the \textit{salesFigures} refer to the number of products sold.
Thus, the \textit{customerSatisfaction} is designed to directly influence potential customers' interest in a particular product. The higher the \textit{customerSatisfaction} in general, the more customers are willing to pay for exactly the same product. 
In this game, \textit{customerSatisfaction} is composed of various variables from almost all parts of the game, which are listed below:
\begin{enumerate}
      \item $averageOverallAppeal_{cS}$ ($aOA_{cS}$)
      \item \textit{totalSupportQuality (tSQ)}
      \item \textit{logisticIndex (lI)}
      \item \textit{companyImage (cI)}
      \item \textit{employerBranding (eB)}
\end{enumerate}
Each of these variables has a weighted influence on the calculation of the customer satisfaction, which depends on its importance.
The $averageOverallAppeal_{cS} $ will be explained in the following, as it was not used earlier in the game mechanics like all the other variables used for the calculation of the \textit{customerSatisfaction}.

First, the \textit{overallAppeal} (\textit{\gls{oA}}) is calculated as shown in equation \ref{oA}. Thereby, the result of the \textit{productAppeal} is mutliplied by the \textit{priceAppeal}. 

\begin{equation}
\label{oA}
\begin{aligned}
overallAppeal = productAppeal \cdot priceAppeal
\end{aligned}
\end{equation}

In order to understand equation \ref{oA}, the \textit{productAppeal} and \textit{priceAppeal} must be defined.

The \textit{productAppeal} is the \textit{totalProductQuality} (\textit{\gls{tPQ}}) devided by the \textit{proxyQuality}.
The \textit{proxyQuality} represents the maximum of the \textit{marketProductUtility} or the \textit{highestProductQuality} (of the player's own products) regarding the same \textit{productCategory}.
In turn, the \textit{marketProductUtility} (see equation \ref{marketProductUtility}) depends on the \textit{time} ($t_y$) and on the \textit{productCategory} (\textit{\gls{pC}}). The \textit{marketProductUtility} then represents the sum of all \textit{baseUtilites} of the most recent level of all components in a specific \textit{productCategory} multiplied with a constant $a \in [0.7, 1.2]$. 
% Or a not constant? Changing difficulty
This constant basically determines the production process productivity of the market. Thus, it can be interpreted as the skill of the competitors and therefore, to a large extent, determines the difficulty of the game.

\begin{equation}
\label{marketProductUtility}
\begin{aligned}
    & marketProductUtility(t_y, productCategory) \\
    & = \sum_{cp \in C_{pC}} \argmax\limits_{\{c \in components_{cp}| availabilityDate_c \leq t_y\}}(baseUtility_c)\\
\end{aligned}    
\end{equation}

\begin{equation}
\label{highestProductQuality}
\begin{aligned}
    & highestProductQuality(productCategory) \\
    & = \sum_{p' \neq p \in P} \argmax\limits_{c \in components_p}(baseUtility_c) \\
\end{aligned}    
\end{equation}

\begin{equation}
\label{proxyQuality}
\begin{aligned}
    & proxyQuality(productCategory) \\
    & = \max(marketProductUtility, highestProductQuality)
\end{aligned}    
\end{equation}

\begin{equation}
    productAppeal = \dfrac{tPQ}{proxyQuality}
\end{equation}

Moreover, the \textit{priceAppeal} for a product is the ratio of the \textit{proxyPrice} and the \textit{salesPrice} (set by the player). The \textit{proxyPrice} in turn represents the sum of the \textit{baseCost} of all components of a product with the same \textit{componentLevel}. So, for example, if $proxyPrice = 60$ and $salesPrice = 100$, the \textit{priceAppeal} would be $0.6$.
%Steffen will make the equation nicer
% Probably not needed. We can just choose a constant for each product type.
\begin{equation}
    proxyPrice = \sum_{cp \in C_{cP}} baseCost_{c} with the same componentLevel
\end{equation}
\begin{equation}
    priceAppeal = \dfrac{proxyPrice}{salesPrice}
\end{equation}

Furthermore, the \textit{averageOverallAppeal} (\textit{\gls{aOA}}) of all products must be determined, as there is only one \textit{customerSatisfaction} calculated. That means, that the \textit{customerSatisfaction} is not calculated for every single product, but has the same influence for each product in the demand calculation, which is explained in more detail in the following chapter \ref{demandCalc}.
\begin{equation}
    averageOverallAppeal = \frac{\sum_{p \in P} oA}{|P|}
\end{equation}

Afterwards, the $aOA_{cS}$, which is needed for the calculation of the \textit{customerSatisfaction}, can be calculated using a sigmoid function, as shown in equation \ref{aOA_cS}. This ensures that the upper value range of the $aOA_{cS}$ is kept by $1$.
Additionally, within the sigmoid function, \textit{aOA} is multiplicated by $0.7$. This way, the $aOA_{cS}$ would be about $60\%$ for an \textit{averageOverallAppeal} of $1$, about $89\%$ for an \textit{averageOverallAppeal} of $2$, and about $34\%$ for an \textit{averageOverallAppeal} of $0.5$. An \textit{averageOverallAppeal} of $1$ means that the player's products have on average the same appeal as the competitor's products. So, a value of $2$ means that the player's products have on average an appeal, which is twice as good than the competitor's products. A value of $3$ means that the player's products have on average an appeal, that is half as good as the competitor's products. 
Apart from that, the result of the sigmoid function must be multiplied by 100 to ensure that the \textit{averageOverallAppeal} has the same influence on the calculation of the \textit{customerSatisfaction} as all other variables (tSQ, lI, cI, eB). This way, the value range of each variable that influences the \textit{customerSatisfaction} goes from $0$ to $100$.

\begin{equation}
\label{aOA_cS}
   aOA_{cS}(aOA) = tanh(aOA \cdot 0.7) \cdot 100
\end{equation}

Finally, the \textit{customerSatisfaction} is calculated as shown in equation \ref{cSCalc}. This calculation is based on a sigmoid function and has a threshold of $1$, as the \textit{customerSatisfaction} cannot become larger than 1, which equals 100\%.

\begin{equation}
\begin{aligned}
\label{cSCalc}
    & cS(aOA_{cS}, tSQ, lI, cI, eB) \\
    & = tanh(\dfrac{(0.5 \cdot aOA_{cS} + 0.2 \cdot tSQ + 0.1 \cdot lI + 0.15 \cdot cI + 0.05 \cdot eB)}{100})
\end{aligned}   
\end{equation}



\begin{comment}
Moreover, there are three main levels of customer requirements regarding products: Must haves, satisfiers and delighters. \cite{krienke_messung_2009}
Must haves are the bare minimum requirements expected of customers. The customers do not show exceptional appreciation for the must haves, but if they are not met, the customer will show dissatisfaction. Satisfiers are the requirements that the customer expressly wishes. If you offer better or more of these satisfiers, then the customers will appreciate it more and be more satisfied. Delighters are the extras or the add-ons. The lack of these characteristics will not make the customer dissatisfied but adding these would greatly increase the customer's satisfaction. In our game, these three levels of customer requirements are included by the following calculation of the customer satisfaction, which depends on a product's \textit{totalProductQuality}. This means, if the user chooses newer, better components for producing a product, then the product's \textit{totalProductQuality} will increase, which again influences the calculation of the customer satisfaction.
    \begin{equation}
    \label{cS_Calc}
    \begin{aligned}
    If \; tPQ \leq \ 40: (0,6*tPQ + 0,15*tSQ + 0,1*lI + 0,1*cI + 0,05*eB)\\
    ElseIf \; tPQ \leq \ 60: (0,5*tPQ + 0,2*tSQ + 0,1*lI + 0,15*cI + 0,05*eB)\\
    Else: (0,45*tPQ + 0,25*tSQ + 0,1*lI + 0,15*cI + 0,05*eB)
    \end{aligned}
    \end{equation}
Each line of the calculation \ref{cS_Calc} refers to one of the three main levels of customer requirements. So, the first line of the calculation refers to a product's must haves, the second line refers to satisfiers and the third line refers to the delighters level of the customer requirements.
Although the calculation \ref{cS_Calc} looks quite static, this is not the case as the variables, on which the calculation is based, are not static but change continuously.
\end{comment}
