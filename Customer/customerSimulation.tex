\section{Customer Simulation}
\label{sec:customsim}
\subsection{Customer Satisfaction}
%Maike
%Only one cS, not for every product
%for the calculation of cS we included nearly all parts of our game
In this game, the customer simulation is a weighted combination of the product price and the customer satisfaction (cS).
This means that product price and customer satisfaction are the two key indicators for customers' buying decisions. Thus, these two factors determine the extent to which customers are interested in buying a product or not.\\
How exactly the customer interest is calculated is discussed in detail in chapter \ref{demandCalc}. The prerequisite for this is the calculation of customer satisfaction, which will be discussed in more detail in this chapter.\\
As well as in reality, the customer satisfaction plays an important role in terms of the number of products sold and thus in terms of sales. \cite{deptolla_effects_2004}
Thus, the customer satisfaction is designed to directly influence potential customers' interest in a particular product. The higher the customer satisfaction in general, the more customers are willing to pay for exactly the same product. 
In this game, the customer satisfaction is composed of the variables listed below:
\begin{enumerate}
      \item Satisfaction with the totalProductQuality (tPQ)
      \item Satisfaction with the totalSupportQuality (tSQ)
      \item Satisfaction with the LogisticIndex (lI)
      \item Satisfaction with the companyImage (cI)
      \item Satisfaction with the employerBranding (eB)
\end{enumerate}

\begin{comment}
\begin{enumerate}
      \item Satisfaction with the totalSupportQuality (tSQ), based on
      \begin{enumerate}
         \item totalSupportTypeQuality
         \item totalSupportQualityOfWork
      \end{enumerate}
      \item Satisfaction with the totalProductQuality (tPQ), based on
      \begin{enumerate}
         \item totalComponentQuality
         \item manufactureEfficiency
         \item productionTechnology (Eco-Index)
         \item totalEngineerQualityOfWork
      \end{enumerate}
      \item Satisfaction with the LogisticIndex (lI), based on
      \begin{enumerate}
         \item externalLogisticIndex
         \item internalLogisticIndex
      \end{enumerate}
      \item Satisfaction with the companyImage (cI), based on
      \begin{enumerate}
          \item Marketing campaigns
      \end{enumerate}
      \item Satisfaction with the employerBranding (eB), based on 
      \begin{enumerate}
          \item totalJobSatisfaction
          \item companyImage
      \end{enumerate}
      \item Satisfaction with productionPerformance (pP)
\end{enumerate}
\end{comment}

Each of these variables has a weighted influence on the calculation of the customer satisfaction, which depends on its importance.
Moreover, there are three main levels of customer requirements regarding products: Must haves, satisfiers and delighters. \cite{krienke_messung_2009}
Must haves are the bare minimum requirements expected of customers. The customers do not show exceptional appreciation for the must haves, but if they are not met, the customer will show dissatisfaction. Satisfiers are the requirements that the customer expressly wishes. If you offer better or more of these satisfiers, then the customers will appreciate it more and be more satisfied. Delighters are the extras or the add-ons. The lack of these characteristics will not make the customer dissatisfied but adding these would greatly increase the customer's satisfaction. In our game, these three levels of customer requirements are included by the following calculation of the customer satisfaction, which depends on a product's \textit{totalProductQuality}. This means, if the user chooses newer, better components for producing a product, then the product's \textit{totalProductQuality} will increase, which again influences the calculation of the customer satisfaction.
    \begin{equation}
    \label{cS_Calc}
    \begin{aligned}
    If \; tPQ \leq \ 40: (0,6*tPQ + 0,15*tSQ + 0,1*lI + 0,1*cI + 0,05*eB)\\
    ElseIf \; tPQ \leq \ 60: (0,5*tPQ + 0,2*tSQ + 0,1*lI + 0,15*cI + 0,05*eB)\\
    Else: (0,45*tPQ + 0,25*tSQ + 0,1*lI + 0,15*cI + 0,05*eB)
    \end{aligned}
    \end{equation}

Each line of the calculation \ref{cS_Calc} refers to one of the three main levels of customer requirements. So, the first line of the calculation refers to a product's must haves, the second line refers to satisfiers and the third line refers to the delighters level of the customer requirements.
Although the calculation \ref{cS_Calc} looks quite static, this is not the case as the variables, on which the calculation is based, are not static but change continuously.
    
%Most common types of customer needs regarding products and the main levels of customer requirements: Functionality, price, design, reliability/availability/durability, performance, safety and sustainability\\
