Janine \\
\begin{itemize}
    \item All produced products go directly into the warehouse but it only costs if they are not sold / delivered at the same day they are produced.
    \item For each product in the warehouse at the end of the day (after all packages are delivered according the demand) the company has to pay
    \item If the warehouse is full, it shouldn't’t be possible to produce more products. That means production is limited to the storage capacity! 
    \item Storage capacity is shared between components and products. To make it a bit more realistic: components need 1/2 unit and products 1 unit
    \item number of products in the warehouse forms the basis for the number of products that can be sold
    \item Characteristics 
    \begin{itemize}
        \item can be build / bought or rented
        \item linear depreciation over 25 years, building losses every year AC/25 on value 
        \item capacity of 5000 units
    \end{itemize}
    \item Cost 
    \begin{itemize}
        \item building-costs (one time) | rental cost (monthly)
        \item maintenance costs (yearly or monthly) 
        \item Daily costs for storage = number units * storing cost per unit per day
    \end{itemize}
\end{itemize}