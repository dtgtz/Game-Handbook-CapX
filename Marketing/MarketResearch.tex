\gls{cs} %Customer Satisfaction from glossary
\subsection{Market Research}
%Janine
Market research is the collection and analysis of data with the aim of better understanding the market and customers. This information can be of great advantage for the management of a company. [REFERENCE 1.2 Market Research The Process, Data, and Methods Using Stata]

If a company or an institution wishes to collect data directly from the customer, there are basically two different procedures. The first is to observe customer behaviour and the second is to ask customers directly. This type of data collection leads to so-called primary data. [REFERENCE 4.4 Market Research The Process, Data, and Methods Using Stata]

As part of Capitalism X, we only consider direct customer surveys via various media. Basically there are four different ways to conduct a survey, these are personal interviews (PI), telephone interviews (TI), mail surveys (MS) and online surveys (OS). Table \ref{MR_survey_types_characteristics} shows some important features of surveys and how each survey type performs in them. [REFERENCE 4.4.2.2 Market Research The Process, Data, and Methods Using Stata] 

\begin{table}[ht]
\centering
\begin{tabular}{|l|r|r|r|r|}
\hline
        & PI & TI  & MS & OS \\
explain a complex issue         & ++    & +     & -    & --  \\
demonstrate trial products      & ++    & -     & -    & -   \\
response rate                   & +     & o     & -    & -   \\
influence of the interviewer    & --    & -     & o    & o   \\
possibility of asking questions & ++    & +     & --   & --  \\
length of the field phase       & o     & +     & -    & -   \\
cots                            & --    & -     & +    & ++  \\
quality of the data             & ++    & +     & -    & --  \\
\hline
\end{tabular}
\caption{Characteristics personal survey types}
\label{MR_survey_types_characteristics}
\end{table}

The characteristics of the various data collection methods naturally also affect the results of market research. [REFERENCE 4.4.2.1 Market Research The Process, Data, and Methods Using Stata] For example, it takes longer for an online survey to produce usable results than for a personal interview. Overall, the quality of the data is best through personal surveys, followed by telephone surveys, mail surveys and worst online surveys. Looking at the cost of a survey, it's the other way around, web surveys are the cheapest and personal interviews the most expensive.. 
In principle, the player can decide whether the market research should be carried out by the company itself or by an external market research institute. If he decides to conduct the surveys internally, he can choose between the various methods of primary data collection and thus influence the quality of the results himself.
The quality at this point expresses the extent to which the result data can deviate from the exactly calculated results. If the data collection is carried out by an institute, the data are determined by telephone interviews by default. The percentage influence on the price of a market statistic by the choice of data collection methods, the time between the commission of the survey and the results as well as the impact on the quality of the study can be seen in Table \ref{MR_survey_types_influence}. 

\begin{table}[ht]
\centering
\begin{tabular}{|l|r|r|r|r|}
\hline
        & PI & TI  & MS & OS \\
influence on the price of a statistic     & 150\%    & 125\%     & 110\%     & 100\%   \\
time lack between results                 & 1 day    & 3 days     & 7 days    & 12 days   \\
influence on the data quality             & no errors    & up to +/- 1\%     & up to +/- 2\%    & up to +/- 4\%   \\
\hline
\end{tabular}
\caption{Influence personal survey types}
\label{MR_survey_types_influence}
\end{table}

The player can basically choose between three different market statistics, which are created on the basis of the previously collected data. These statistics cover price sensitivity of customers, customer satisfaction and general market data.
 
The costs for the individual market statistics can be taken from the table \ref{MR_report_price}. The total cost of conducting internal market research is determined by the product of the base price of the respective statistics and the influence on the price of the chosen data collection method. The costs of the external execution correspond to the final costs that the company has to pay.

\begin{table}[ht]
\centering
\begin{tabular}{|l|r|r|}
\hline
        & internal base price  & external price \\
price sensitive report       & 2.500    & 3.200     \\
customer satisfaction report & 2.000    & 2.500     \\
market statistic research    & 1.000    & 1.150     \\
\hline
\end{tabular}
\caption{prices market research reports}
\label{MR_report_price}
\end{table}

The following sections describe the market statistics available to the player in detail.

\subsubsection{Price Sensitive Research}
This report includes a price sensitivity analysis for a previously selected product in the company's product portfolio. The analysis shows how margin, sales number, sales volume, total cost and market share change when the sales price changes. All other factors are kept constant for the calculations. The result is divided into nine price steps, from -20\% to +20\% in 5\% steps. These relate to the current sales price of the product in question. The hypothetical values are calculated as follows:
\begin{equation}
    \begin{aligned}
        for +5\% to +20\%: x = ((100 * new value) / ref value) – 100 \\
        for -5\% to -20\%: x = ((1 - (new value / ref value)) / 100
    \end{aligned}
\end{equation}

Table \ref{MR_price_sensitive} shows the structure of a price sensitive report. 

\begin{table}[ht]
\centering
\begin{tabular}{|l|r|r|r|r|r|r|r|r|r|}
\hline
key figures             & -20\% & -15\% & -10\% & -5\%  & 0\%   & +5\%  & +10\% & +15\%   \\
market price            &       &       &       &       &       &       &       &         \\
product costs           &       &       &       &       &       &       &       &         \\
margin                  &       &       &       &       &       &       &       &         \\
sales number            &       &       &       &       &       &       &       &         \\
sales volume            &       &       &       &       &       &       &       &         \\
market share            &       &       &       &       &       &       &       &         \\
margin change           &       &       &       &       &       &       &       &         \\
market share change     &       &       &       &       &       &       &       &         \\
net change              &       &       &       &       &       &       &       &         \\
\hline
\end{tabular}
\caption{Structure price sensitive report}
\label{MR_price_sensitive}
\end{table}

\subsubsection{Customer Satisfaction Research}
This report allows the player to see an overview of customer satisfaction over the last four quarters. The player can choose whether the result should refer to the entire company, i.e. all products, or only to a specific product. Table \ref{MR_customer_satisfaction} illustrates the structure of the customer satisfaction report. 

\begin{table}[ht]
\centering
\begin{tabular}{|l|r|r|r|r|}
\hline
            & quarter 1   & quarter 2  & quarter 3 & quarter 4 \\
product 1   &             &            &           &           \\
product 1   &             &            &           &           \\
product 1   &             &            &           &           \\
...         &             &            &           &           \\
total       & average     & average    & average   & average   \\
\hline
\end{tabular}
\caption{Structure customer satisfaction report}
\label{MR_customer_satisfaction}
\end{table}

\subsubsection{Market Statistic Research}
This analysis provides various market information on product level. This means that the key figures for the products are provided individually and also the average for the entire company. The report contains the current product price, market share, sales numbers and product quality. Table \ref{MR_market_statistic} is an example for the structure of the report. 

\begin{table}[ht]
\centering
\begin{tabular}{|l|r|r|r|r|}
\hline
                        & product 1   & product 2  & product 3 & ...       \\
product price           &             &            &           &           \\
market share            &             &            &           &           \\
sales numbers           &             &            &           &           \\
product quality         &             &            &           &           \\
service quality         &             &            &           &           \\
customer satisfaction   &             &            &           &           \\
\hline
\end{tabular}
\caption{Structure market statistic report}
\label{MR_market_statistic}
\end{table}