\subsection{Lobbyist} \label{lobbyist_simulation}
%Janine
If the company is in contact with a lobbyist, he tries to mitigate negative government decisions on the company. For its services the lobbyist demands a monthly compensation payment. The effect of the lobbyist is reflected by the tax rate. 
To determine realistic tax rates, corporates selling similar products to the ones in the game were compared. Their average is the standard tax rate of CapitalismX. (REFERENCES)

\begin{table}[]
\centering
\begin{tabular}{|l|l|l|l|l|l|l|}
\hline
Example & IBM & HP Enterprise & Google & Nintendo & Apple & Average \\
Tax Rate & 23\% & 16\% & 12\% & 30\% & 18\% & 20\% \\ \hline
\end{tabular}
\caption{Example Tax rates}
\label{Example_Tax}
\end{table}

Only one lobbyist can be assigned at a time, if the player wishes to hire another lobbyist, he must first fire the current one. Each lobbyist has a name, a type and a price. The type determines how influential the lobbyist is. The player can choose between four different lobbyists: Mayor, Worker’s Union Leader, Congressman and Senator. The Senator is the most influential but also the most expensive lobbyist. Table \ref{influence_lobbyist} shows the relationship between type, influence and price. 

\begin{table}[ht]
\centering
\begin{tabular}{|l|r|r|}
\hline
Type                    & tax rate  & price \\
Mayor                   & 16\%      & 1.000     \\
Worker's Union Leader   & 18\%      & 1.000     \\
Congressman             & 13\%      & 5.000     \\
Senator                 & 10\%      & 10.000     \\
\hline
\end{tabular}
\caption{Influence and price lobbyist}
\label{influence_lobbyist}
\end{table}

In order to find out which lobbyist is the most rewarding for the company, the player must calculate the amount of the tax reduction and compare it with the costs incurred by the lobbyist.
 



 




